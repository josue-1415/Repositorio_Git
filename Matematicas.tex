\documentclass[12pt,a4paper]{article}
\usepackage[utf8]{inputenc}
\usepackage[spanish]{babel}
\usepackage{graphicx}
\usepackage[left=2.54cm,right=2.54cm,top=2cm,bottom=2cm]{geometry}
\author{Josué Daniel Isaula Ávila}
\title{Git Repository}
\begin{document}
\maketitle 

\begin{center}
    \huge Universidad Nacional Autónoma de Honduras \\ 
    \large Valle de Sula \\
    \vspace{5 mm}
    \includegraphics[scale =0.6]{Imagenes/Logo_UNAH.png}
\end{center}
\newpage

\section{Clonando un repositorio existente}
Si deseas obtener una copia de un repositorio Git existente por ejemplo, un
proyecto en el que te gustaría contribuir el comando que necesitas es git clone. Si
estás familizarizado con otros sistemas de control de versiones como Subversion, verás
que el comando es "clone" en vez de "checkout". Es una distinción importante, ya que
Git recibe una copia de casi todos los datos que tiene el servidor. Cada versión de
cada archivo de la historia del proyecto es descargada por defecto cuando ejecutas git
clone. De hecho, si el disco de tu servidor se corrompe, puedes usar cualquiera de los
clones en cualquiera de los clientes para devolver el servidor al estado en el que
estaba cuando fue clonado (puede que pierdas algunos hooks del lado del servidor y
demás, pero toda la información acerca de las versiones estará ahí) véase
Configurando Git en un servidor para más detalles. \vspace{0.5 cm}

\centering 
\includegraphics[scale = 0.5]{Imagenes/Honduras.png}

\newpage 
\begin{enumerate}
    \item Honduras 
    \item El Salvador 
    \item Costa Rica 
    \item Nicaragua
    \item Guatemala
    \item Panama
    \item Belice 
\end{enumerate}

\Huge{\bf Bitcoin} \\
\vspace{0.5 cm}
\centering
\includegraphics[scale = 0.3]{Imagenes/bitcoin.jpg}

\end{document}